\begin{center}
  \Large
  \textbf{KATA PENGANTAR}
\end{center}

\addcontentsline{toc}{chapter}{KATA PENGANTAR}

\vspace{2ex}

% Ubah paragraf-paragraf berikut dengan isi dari kata pengantar

Puji dan syukur kehadirat Tuhan Yang Maha Esa,karena atas berkatnya penulis dapat menyelesaikan penelitian ini. Penelitian ini disusun dalam rangka memenuhi syarat kelulusan yang ada pada Program Studi S-1 Teknik Komputer Departemen Teknik Komputer Fakultas Teknologi Elektro dan Informatika Cerdas Institut Teknologi Sepuluh Nopember. Tentunya penulis dapat menyelesaikan penelitian ini tidak hanya karena kemampuan pribadi penulis. Namun, berbagai pihak ikut turut membantu penulis dalam menyelesaikan penelitian ini. Oleh karena itu penulis mengucapkan banyak terimakasih kepada :

\begin{enumerate}
  \item{}
  Bapak Ahmad Zaini, S.T., M.T selaku dosen pembimbing I yang memberikan masukan,arahan mengenai penelitian ini serta ilmu moral dalam menyelesaikan penelitian kepada penulis.
  \item{}
  Bapak Eko Pramunanto S.T., M.T selaku dosen pembimbing II yang juga turut memberikan banyak masukan, arahan mengenai penelitian dan dalam penulisan buku serta ilmu moral dalam menyelesaikan penelitian kepada penulis.
  \item{}
  Ibu, Ayah, Kakak, dan Adik saya 
  
\end{enumerate}
Penulis menyadari bahwa sesuatu yang sempurna hanya milik Tuhan. Oleh karena itu, penulis sangat terbuka apabila ada kritik dan saran untuk kekurangan penelitian ini. Akhir kata, semoga penelitian ini bisa berguna bagi banyak pihak.

\begin{flushright}
  \begin{tabular}[b]{c}
    \place{}, \MONTH{} \the\year{} \\
    \\
    \\
    \\
    \\
    \name{}
  \end{tabular}
\end{flushright}
