\begin{center}
  \large\textbf{ABSTRAK}
\end{center}

\addcontentsline{toc}{chapter}{ABSTRAK}

\vspace{2ex}

\begingroup
% Menghilangkan padding
\setlength{\tabcolsep}{0pt}

\noindent
\begin{tabularx}{\textwidth}{l >{\centering}m{2em} X}
  Nama Mahasiswa    & : & \name{}         \\

  Judul Tugas Akhir & : & \tatitle{}      \\

  Pembimbing        & : & 1. \advisor{}   \\
                    &   & 2. \coadvisor{} \\
\end{tabularx}
\endgroup

% Ubah paragraf berikut dengan abstrak dari tugas akhir
Seseorang tentunya membutuhkan konsentrasi dalam melakukan pekerjaan. Kantuk seringkali menjadi masalah yang menggangu konsentrasi seseorang dalam melakukan pekerjaan sehingga dapat menyebabkan kecelakaan kerja. Penelitian ini bertujuan untuk mengklasifikasikan tingkat kantuk seseorang menggunakan metrik PERCLOS (Persentase Penutupan Mata) dan MAR (Nilai Bukaan Mulut) sebagai indikator kantuk. Penelitian ini menggunakan Skala Karolinska untuk menilai subjektivitas kantuk dan menerapkan Support Vector Machine (SVM) untuk klasifikasi otomatis. Penelitian dilakukan dengan menggunakan dataset video yang telah terverifikasi. Melalui dataset yang sudah dilatih,dikumpulkan data PERCLOS dan MAR. SVM dilatih dengan data ini untuk memprediksi tingkat kantuk yang dimiliki oleh video. Diharapkan hasil eksperimen menunjukkan bahwa gabungan PERCLOS dan MAR efektif untuk mendeteksi kantuk dengan akurasi yang signifikan. Hal itu diharapkan menunjukkan potensi metode ini untuk diimplementasikan dalam sistem keselamatan kerja cerdas. Selain itu, analisis ini diharapkan dapat memberikan pemahaman baru bagi pembuat kebijakan dan dapat membuka penelitian yang lain untuk mengembangkan teknologi pencegahan kantuk yang lebih baik, sehingga dapat mengurangi risiko kecelakaan akibat kantuk seseorang.

% Ubah kata-kata berikut dengan kata kunci dari tugas akhir
\vspace{2ex}
\noindent
\textbf{Kata Kunci: \emph{Kantuk, PERCLOS, MAR, Skala Karolinska, SVM, Model}}