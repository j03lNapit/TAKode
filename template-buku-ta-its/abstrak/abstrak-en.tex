\begin{center}
  \large\textbf{ABSTRACT}
\end{center}

\addcontentsline{toc}{chapter}{ABSTRACT}

\vspace{2ex}

\begingroup
% Menghilangkan padding
\setlength{\tabcolsep}{0pt}

\noindent
\begin{tabularx}{\textwidth}{l >{\centering}m{3em} X}
  \emph{Name}     & : & \name{}         \\

  \emph{Title}    & : & \engtatitle{}   \\

  \emph{Advisors} & : & 1. \advisor{}   \\
                  &   & 2. \coadvisor{} \\
\end{tabularx}
\endgroup

% Ubah paragraf berikut dengan abstrak dari tugas akhir dalam Bahasa Inggris
\emph{An individual certainly needs concentration to perform work. Drowsiness often becomes a problem that disrupts a person's concentration while working, which can lead to work accidents. This study aims to classify the level of drowsiness in a person using the PERCLOS (Percentage of Eye Closure) and MAR (Mouth Opening Value) metrics as indicators of drowsiness. This research uses the Karolinska Scale to assess the subjectivity of drowsiness and applies a Support Vector Machine (SVM) for automatic classification. The study was conducted using a verified video dataset. Through this trained dataset, PERCLOS and MAR data were collected. The SVM was trained with this data to predict the level of drowsiness of the video. It is expected that the experimental results will show that the combination of PERCLOS and MAR is effective for detecting drowsiness with significant accuracy. This is expected to demonstrate the potential of this method for implementation in intelligent work safety systems. Furthermore, this analysis is expected to provide new insights for policymakers and open up further research to develop better drowsiness prevention technology, thus reducing the risk of accidents caused by an individual's drowsiness.}

% Ubah kata-kata berikut dengan kata kunci dari tugas akhir dalam Bahasa Inggris
\vspace{2ex}
\noindent
\textbf{Keywords: \emph{Drowsiness, PERCLOS, MAR, Karolinska Scale, SVM, Model}}
