% Atur variabel berikut sesuai namanya

% nama
\newcommand{\name}{Joel Napitupulu}
\newcommand{\authorname}{Napitupulu, Joel}
\newcommand{\nickname}{Joe}
\newcommand{\advisor}{Ahmad Zaini, S.T., M.T}
\newcommand{\coadvisor}{Eko Pramunanto, S.T., M.T}
\newcommand{\examinerone}{Prof. Dr. Ir. Mauridhi Hery P., M.Eng.}
\newcommand{\examinertwo}{Ir. Hanny Budinugroho, S.T., M.T.}
\newcommand{\examinerthree}{Muhtadin, S.T., M.T.}
\newcommand{\headofdepartment}{Dr. Supeno Mardi Susiki Nugroho,, S.T., M.T}

% identitas
\newcommand{\nrp}{5024201017}
\newcommand{\advisornip}{19750419200212 1 003}
\newcommand{\coadvisornip}{19661203199412 1 001}
\newcommand{\examineronenip}{19580916198601 1 001}
\newcommand{\examinertwonip}{19610706198701 1 001}
\newcommand{\examinerthreenip}{19810609200912 1 003}
\newcommand{\headofdepartmentnip}{19700313199512 1 001}

% judul
\newcommand{\tatitle}{KLASIFIKASI KANTUK BERDASARKAN NILAI BUKAAN MATA \emph{(PERCLOS)} DAN
NILAI BUKAAN MULUT \emph{(MAR)} MENGGUNAKAN SKALA KAROLINSKA DAN
SUPPORT VECTOR MACHINE}
\newcommand{\engtatitle}{DROWSINESS CLASSIFICATION BASED ON PERCENTAGE OF EYELID CLOSURE  \emph{(PERCLOS)} AND MOUTH ASPECT RATIO \emph{(MAR)} USING KAROLINSKA SCALE AND SUPPORT VECTOR MACHINE}

% tempat
\newcommand{\place}{Surabaya}

% jurusan
\newcommand{\studyprogram}{Teknik Komputer}
\newcommand{\engstudyprogram}{Computer Engineering}

% fakultas
\newcommand{\faculty}{Teknologi Elektro Dan Informatika Cerdas}
\newcommand{\engfaculty}{Intelligent Electrical and Informatics Technology}

% singkatan fakultas
\newcommand{\facultyshort}{FTEIC}
\newcommand{\engfacultyshort}{ELECTICS}

% departemen
\newcommand{\department}{Teknik Komputer}
\newcommand{\engdepartment}{Computer Engineering}

% kode mata kuliah
\newcommand{\coursecode}{EC234801}
